\documentclass[10pt,a4paper]{article}
\usepackage{graphicx}
\usepackage{amssymb}
\usepackage{fullpage}
\usepackage{amsmath}
\usepackage{url}

\title{Senior Astrophysics 2011 Assignment 2\footnote{Updates can be found at \texttt{https://github.com/goldfish-dave/SeniorAstroAssignment2}}}

\date{}
\author{D. G. Wilcox \\
		309248035}

\begin{document}
\maketitle

\section*{Question 1}
\begin{itemize}
    \item[(a)]
    \item[(b)]
    \item[(c)]
    \item[(d)]
    \item[(e)]
\end{itemize}
\section*{Question 2}
% Reaction Equations
\newcommand{\reactionOne}{$p + p \rightarrow ^{2}H + e^{+} + \nu_{e}$}
\newcommand{\reactionTwo}{$^{2}H + p \rightarrow ^{3}He + \gamma$}
\newcommand{\reactionThree}{$^{3}He + ^{3}He \rightarrow ^{4}He + 2p$}

% Q-values
\newcommand{\qvalueOne}{$+2.73 \times 10^{5}$}
\newcommand{\qvalueTwo}{$-1.54 \times 10^{7}$}
\newcommand{\qvalueThree}{$-4.24 \times 10^{7}$}

\begin{table}[h]
    \begin{tabular}{c|c|c|c|c}
        \#   &   Reaction            &   LHS Mass   (u)    &   RHS Mass (u)    &   Q-value (MeV) \\
        \hline \\
        1    &   \reactionOne{}      &   2.0145530         &   2.0146486       &   \qvalueOne{}  \\
        2    &   \reactionTwo{}      &   3.0213765         &   3.0160000       &   \qvalueTwo{}  \\
        3    &   \reactionThree{}    &   6.0320000         &   6.0171550       &   \qvalueThree{} \\
    \end{tabular}
\end{table}
Since the entire PP-1 chain results in $2(\mbox{Reaction } 1) + 2(\mbox{Reaction } 2) \rightarrow \mbox{Reaction } 3$, if we sub in our values and subtract the left from the right we find that the Q-value for the PP-1 chain is $1.22 \times 10^{7}$ MeV, which is in close agreement with the tabulated 12.86 Mev\footnote{\url{http://en.wikipedia.org/wiki/Proton-proton_chain_reaction#The_pp_I_branch}}.
\section*{Question 3}
\section*{Question 4}
\begin{itemize}
    \item[(a)]
    \item[(b)]
    \item[(c)]
\end{itemize}

\end{document}
